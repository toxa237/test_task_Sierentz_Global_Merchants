% !TeX program = lualatex							    	
% !TeX encoding = utf8
% !TeX spellcheck = uk_UA

\documentclass[14pt]{extarticle}
\usepackage{amsmath,amsthm,amssymb}
\usepackage{mathtext}
\usepackage{fontspec}
\setsansfont{CMU Sans Serif}%{Arial}
\setmainfont{CMU Serif}%{Times New Roman}
\setmonofont{CMU Typewriter Text}%{Consolas}
\defaultfontfeatures{Ligatures={TeX}}
\usepackage[math-style=TeX]{unicode-math}
\usepackage[english, russian, ukrainian]{babel}
\usepackage{misccorr}
\usepackage{textcomp}
\usepackage[absolute]{textpos}
\usepackage{wrapfig}
\usepackage{graphicx}     
\usepackage{multirow}
\usepackage{indentfirst}
\usepackage{microtype} 
\usepackage{tikz}
\usepackage{pgfplots}
\pgfplotsset{compat=newest}
\usepackage{pgfplotstable}
\usepackage{booktabs}
\usepackage{array}
\usepackage{listings}

\usepackage{colortbl}
\usepackage[%
a4paper,%
footskip=1cm,%
headsep=1cm,% 
top=2cm, %поле сверху
bottom=2cm, %поле снизу
left=2cm, %поле ліворуч
right=2cm, %поле праворуч
]{geometry}

\usepackage[%colorlinks=true,
%urlcolor = blue, %Colour for external hyperlinks
%linkcolor  = malina, %Colour of internal links
citecolor  = green, %Colour of citations
bookmarks = true,
bookmarksnumbered=true,
unicode,
linktoc = all,
hypertexnames=false,
pdftoolbar=false,
pdfpagelayout=TwoPageRight,
pdfauthor={student}]{hyperref}

\definecolor{codegreen}{rgb}{0,0.6,0}
\definecolor{codegray}{rgb}{0.5,0.5,0.5}
\definecolor{codepurple}{rgb}{0.58,0,0.82}
\definecolor{backcolour}{rgb}{0.95,0.95,0.92}
\lstdefinestyle{mystyle}{
	backgroundcolor=\color{backcolour},  
	commentstyle=\color{codegreen},
	keywordstyle=\color{magenta},
	numberstyle=\tiny\color{codegray},
	stringstyle=\color{codepurple},
	basicstyle=\ttfamily, %\footnotesize
	breakatwhitespace=false,
	breaklines=true,                 
	captionpos=b,                    
	keepspaces=true,                 
	numbers=left,                    
	numbersep=5pt,                  
	showspaces=false,                
	showstringspaces=false,
	showtabs=false,                  
	tabsize=2
}
\lstset{style=mystyle}

\begin{document}

\begin{center}
Тестове завдання
\end{center}

\section*{1}
Consider influenza epidemics for 2-person families. The probability is 21\% that at least one has disease. The probability that the husband has contracted influenza is 15\% while the probability that both the wife and husband have contracted the disease is 10\%. What is the probability that the wife has influenza?\\

Застосуємо формулу для ймовірності появи хоча б одної з 2 подій

\begin{equation*}
P(A\cup B)=P(A)+P(B)-P(AB)
\end{equation*}

Тут $ P(A\cup B) $ це ймовірність що хоча б 1 з двох є хворим, $ P(A) $ ймовірність що хворий чоловік, $ P(AB) $ що хворі обоє та $ P(B) $ що хвора дружина. Тоді ймовірність того що хвора дружина:

\begin{equation*}
P(B) = P(A\cup B) - P(A) + P(AB) = 16\%
\end{equation*}

\section*{2}

1. How many tons worth of fruit does an average seller have?
\begin{lstlisting}[language=SQL]
SELECT seller_id, AVG(fruit_weight)
FROM seller_info
GROUP BY seller_id;
\end{lstlisting}

2. How many sellers have at least one client who purchased their fruit?
\begin{lstlisting}[language=SQL]
WITH t as(
SELECT si.seller_id, COUNT(ci.fruit_id)
FROM seller_info as si
LEFT JOIN consumption_info as ci
ON si.seller_id = ci.seller_id
GROUP BY seller_id
HAVING  COUNT(ci.fruit_id) != 0
)
select COUNT(*)
from t;
\end{lstlisting}

\section*{3 та 5}
Там наче питань не було, але не знав в якому вигляді зробити вхід даних, тому там просто функції та вивід прикладів які були у файлі.

\section*{4}
Тут є декілька моментів:
\begin{enumerate}
\item я не зміг імпортувати нормально дані тому я перевів їх в формат \textit{.csv}.
\item Перетворення дат визвало деякі запитання, я одразу додав до року $ 1900 $ та перейменував колонки. Після чого переводив в формати дати. Тому завдання які йдуть дальше я пропустив. Там звісно можна задати наприклад 60р, але прийшлось б писати зайвий код (datetime.datetime.strptime('0060 1 1', '\%Y \%m \%d')).
\item Те питання що я Вам писав на джині. \textit{"Downsample the record to a yearly frequency for each location"}. Що тут потрібно було зробити? Я так зрозумів що потрібно було згрупувати по роках, місяцях і тд. Це можна зробити за допомогою методу \href{https://pandas.pydata.org/docs/reference/api/pandas.DataFrame.resample.html}{"resample"}. Але він в кінці простить вказати що рахувати за цей період, я там вибрав середнє значення, але цього не було вказано в завданні.
\item Деякі речі описані в самому файлі, просто запустіть його. Дані в форматі \textit{.csv} будуть в архіві.
\end{enumerate}
Але все ж таки це завдання хотілось б обговорити.


\end{document}


